\section{Протокол BMQV}\label{MQV}

\subsection{Сообщения}\label{MQV.Messages}

В протоколе BMQV
cтороны пересылают друг другу следующие сообщения:

M0 ($A\to B$): 
$\mathimplicit{\hello_A}$;

M1 ($B\to A$): 
$\mathimplicit{\hello_B\parallel}\ 
\mathimplicit{\Cert(Id_B, Q_B)\parallel}\ 
\langle V_B\rangle_{4\ell}$;

M2 ($A\to B$): 
$\mathimplicit{\Cert(Id_A,Q_A)\parallel}\ 
\langle V_A\rangle_{4\ell}\ \mathoptional{\parallel T_A}$;

M3 ($B\to A$): 
$\mathoptional{T_B}$.

\subsection{Шаги}\label{MQV.Steps}

Формирование общего ключа состоит в выполнении шагов, определенных ниже. При
ошибке на любом из шагов, в том числе при отрицательном результате любой
проверки, протокол прекращается и возвращается признак~\texttt{ОШИБКА}.

\begin{enumerate}
\item
Сторона $A$:
\begin{enumerate}
\item
\implicit{отправляет сообщение~M0}.
\end{enumerate}

\item
Сторона~$B$:
\begin{enumerate}
\item
\implicit{получает сообщение~M0};
\item
$u_B\stackrel{R}{\leftarrow}\{1,2,\ldots,q-1\}$
(в соответствии с требованиями~\ref{COMMON.Ephemeral});
\item
$V_B\leftarrow u_B G$;
\item
отправляет сообщение~M1.
\end{enumerate}
\item
Сторона $A$:
\begin{enumerate}
\item
получает сообщение~M1;
\item
\implicit{проверяет $\Cert(Id_B,Q_B)$};
\item
проверяет, что $V_B\in E_{a,b}^*(\FF_p)$;
\item
$u_A\stackrel{R}{\leftarrow}\{1,2,\ldots,q-1\}$
(в соответствии с требованиями~\ref{COMMON.Ephemeral});
\item
$V_A\leftarrow u_A G$;
\item
$t\leftarrow\bigl\langle
\algname{belt-hash}
(\langle V_A\rangle_{2\ell}\parallel\langle V_B\rangle_{2\ell})
\bigr\rangle_\ell$;
\item
$s_A\leftarrow (u_A-(2^\ell+\overline{t})d_A)\bmod q$;
\item
$K\leftarrow s_A(V_B-(2^\ell+\overline{t})Q_B)$;
\item
если $K=O$, то $K\leftarrow G$;
\item
$K_0\leftarrow\algname{bake-kdf}
(\langle K\rangle_{2\ell},
\Cert(Id_A,Q_A)\parallel \Cert(Id_B,Q_B)\parallel 
\hello_A\parallel\hello_B,
0)$;
\item
$\mathoptional{
K_1\leftarrow\algname{bake-kdf}
(\langle K\rangle_{2\ell},
\Cert(Id_A,Q_A)\parallel \Cert(Id_B,Q_B)\parallel 
\hello_A\parallel\hello_B,
1)}$;
\item
$\mathoptional{T_A\leftarrow\algname{belt-mac}(0^{128},K_1)}$;
\item
отправляет сообщение~M2.
\end{enumerate}

\item
Сторона $B$:
\begin{enumerate}
\item
получает сообщение~M2;
\item
\implicit{проверяет $\Cert(Id_A,Q_A)$};
\item
проверяет, что $V_A\in E_{a,b}^*(\FF_p)$;
\item
$t\leftarrow\bigl\langle
\algname{belt-hash}
(\langle V_A\rangle_{2\ell}\parallel\langle V_B\rangle_{2\ell})
\bigr\rangle_\ell$;
\item
$s_B\leftarrow (u_B-(2^\ell+\overline{t})d_B)\bmod q$;
\item
$K\leftarrow s_B(V_A-(2^\ell+\overline{t})Q_A)$;
\item
если $K=O$, то $K\leftarrow G$;
\item
$K_0\leftarrow\algname{bake-kdf}
(\langle K\rangle_{2\ell},
\Cert(Id_A,Q_A)\parallel \Cert(Id_B,Q_B)\parallel 
\hello_A\parallel\hello_B,
0)$;
\item
$\mathoptional{
K_1\leftarrow\algname{bake-kdf}
(\langle K\rangle_{2\ell},
\Cert(Id_A,Q_A)\parallel \Cert(Id_B,Q_B)\parallel 
\hello_A\parallel\hello_B,
1)}$;
\item
\optional{проверяет, что $T_A=\algname{belt-mac}(0^{128},K_1)$};
\item
$\mathoptional{T_B\leftarrow\algname{belt-mac}(1^{128},K_1)}$;
\item
\optional{отправляет сообщение~M3}.
\end{enumerate}

\item
Сторона $A$:
\begin{enumerate}
\item
\optional{получает сообщение~M3};
\item
\optional{проверяет, что $T_B=\algname{belt-mac}(1^{128},K_1)$}.
\end{enumerate}
\end{enumerate}

Успешное выполнение всех шагов протокола означает, что 
стороны выработали общий ключ~$K_0$.
%
Если при выполнении протокола  
создавалась и обрабатывалась имитовставка~$T_A$ ($T_B$),
то успешное завершение дополнительно означает,
что сторона~$A$ ($B$) подтвердила свой ключ и, таким образом, 
прошла аутентификацию перед противоположной стороной.

