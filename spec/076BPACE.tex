\section{Протокол BPACE}\label{PACE}

\subsection{Cообщения}\label{PACE.Messages}

Стороны пересылают друг другу следующие сообщения:

M0 ($A\to B$): 
$\mathimplicit{\hello_A}$;

M1 ($B\to A$): 
$\mathimplicit{\hello_B\parallel}\ 
Y_B$;

M2 ($A\to B$): 
$Y_A\parallel\langle V_A\rangle_{4l}$;

M3 ($B\to A$): 
$\langle V_B\rangle_{4l}\ \mathoptional{\parallel T_B}$;

M4 ($A\to B$): 
$\mathoptional{T_A}$.

\subsection{Шаги протокола}\label{PACE.Steps}

Формирование общего ключа состоит в выполнении шагов,
определенных ниже. При ошибке на любом из шагов,
в том числе при отрицательном результате любой проверки, 
протокол прекращается и возвращается признак~\texttt{ОШИБКА}.

\begin{enumerate}
\item
Сторона $A$:
\begin{enumerate}
\item
\implicit{отправляет сообщение M0}.
\end{enumerate}

\item
Сторона~$B$:
\begin{enumerate}
\item
\implicit{получает сообщение M0};
\item
$R_B\stackrel{R}{\leftarrow}\{0,1\}^{l}$ 
(в соответствии с требованиями~\ref{COMMON.Ephemeral});
\item
$K_2\leftarrow\algname{belt-hash}(P)$;
\item
$Y_B\leftarrow \texttt{belt-ecb}(R_B,K_2)$;
\item
отправляет сообщение M1.
\end{enumerate}

\item
Сторона $A$:
\begin{enumerate}
\item
получает сообщение M1;
\item
проверяет, что $|Y_B|=l$;
\item
$K_2\leftarrow\algname{belt-hash}(P)$;
\item
$R_B\leftarrow \algname{belt-ecb}^{-1}(Y_B,K_2)$;
\item
$R_A\stackrel{R}{\leftarrow}\{0,1\}^{l}$
(в соответствии с требованиями~\ref{COMMON.Ephemeral});
\item
$Y_A\leftarrow\algname{belt-ecb}(R_A,K_2)$;
\item
$W\leftarrow\algname{bake-swu}(p,a,b,R_A\parallel R_B)$;
\item
$u_A\stackrel{R}{\leftarrow}\{1,2,\ldots,q-1\}$
(в соответствии с требованиями~\ref{COMMON.Ephemeral});
\item
$V_A\leftarrow u_A W$;
\item
отправляет сообщение M2.
\end{enumerate}

\item
Сторона $B$:
\begin{enumerate}
\item
получает сообщение M2;
\item
проверяет, что~$V_A\in E_{a,b}^*(\FF_p)$;
\item
проверяет, что $|Y_A|=l$;
\item
$R_A\leftarrow \algname{belt-ecb}^{-1}(Y_A,K_2)$;
\item
$W\leftarrow\algname{bake-swu}(p,a,b,R_A\parallel R_B)$;
\item
$u_B\stackrel{R}{\leftarrow}\{1,2,\ldots,q-1\}$
(в соответствии с требованиями~\ref{COMMON.Ephemeral});
\item
$V_B\leftarrow u_B W$;
\item
$K\leftarrow u_B V_A$;
\item
$K_0\leftarrow\algname{bake-kdf}
(\langle K\rangle_{2l},
\langle V_A\rangle_{2l}\parallel
\langle V_B\rangle_{2l}\parallel
\hello_A\parallel\hello_B,
0)$;
\item
$\mathoptional{K_1\leftarrow\algname{bake-kdf}
(\langle K\rangle_{2l},
\langle V_A\rangle_{2l}\parallel
\langle V_B\rangle_{2l}\parallel
\hello_A\parallel\hello_B,
1)}$;
\item
$\mathoptional{T_B\leftarrow\algname{belt-mac}(1^{128}, K_1)}$;
\item
отправляет сообщение M3.
\end{enumerate}

\item
Сторона $A$:
\begin{enumerate}
\item
получает сообщение M3;
\item
проверяет, что~$V_B\in E_{a,b}^*(\FF_p)$;
\item
$K\leftarrow u_A V_B$;
\item
$K_0\leftarrow\algname{bake-kdf}
(\langle K\rangle_{2l},
\langle V_A\rangle_{2l}\parallel
\langle V_B\rangle_{2l}\parallel
\hello_A\parallel\hello_B,
0)$;
\item
$\mathoptional{K_1\leftarrow\algname{bake-kdf}
(\langle K\rangle_{2l},
\langle V_A\rangle_{2l}\parallel
\langle V_B\rangle_{2l}\parallel
\hello_A\parallel\hello_B,
1)}$;
\item
\optional{проверяет, что $T_B=\texttt{belt-mac}(1^{128}, K_1)$};
\item
$\mathoptional{T_A\leftarrow\algname{belt-mac}(0^{128}, K_1)}$;
\item
\optional{отправляет сообщение M4}.
\end{enumerate}

\item
Сторона $B$:
\begin{enumerate}
\item
\optional{получает сообщение M4};
\item
\optional{проверяет, что $T_A=\algname{belt-mac}(0^{128}, K_1)$}.
\end{enumerate}
\end{enumerate}

Успешное выполнение всех шагов протокола означает, что 
стороны выработали общий ключ~$K_0$.
%
Если при выполнении протокола  
создавалась и обрабатывалась имитовставка~$T_A$ ($T_B$),
то успешное завершение дополнительно означает,
что сторона~$A$ ($B$) подтвердила свой ключ и, таким образом, 
прошла аутентификацию перед другой стороной.
