\clearpage
\begin{thebibliography}{999}
\bibitem{LidNid88}
Лидл~Р., Нидеррайтер~Г. Конечные поля\\
{\small М.: Мир, 1988}

\bibitem{ECC}
Hankerson~D., Menezes~A., Vanstone S.
Guide to Elliptic Curve Cryptography\\
{\small N.~Y.: Springer, 2004}\\
{\small (Введение в криптографию на эллиптических кривых)}

\bibitem{MQV}
Law~L., Menezes~A., Qu~M., Solinas~J., Vanstone~S.
An Efficient Protocol for Authenticated Key Agreement\\
{\small Designs, Codes and Cryptography, 28(2), 119-134, 2003}\\
{\small (Эффективный протокол формирования общего ключа и аутентификации)}

\bibitem{STS}
Diffie~W., Oorschot~P., Wiener~M. 
Authentication and Authenticated Key Exchanges\\
{\small Designs, Codes and Cryptography, 2(2): 107--125, 1992}\\
{\small (Аутентификация и обмен ключами с аутентификацией)}

\bibitem{PACE}
Bender~J., Fischlin~M., Kuegler~D. 
Security Analysis of the PACE Key-Agreement Protocol\\
{\small Cryptology ePrint Archive, Report 2009/624, 2009}\\
%Avail. at: \url{eprint.iacr.org/2009/624.pdf}.
{\small (Оценка надежности протокола формирования общего ключа PACE)}

\bibitem{SWU}
Brier~E., Coron~J., Icart~T., Madore~D., Randriam~H., Tibouch~M. 
Efficient Indifferentiable Hashing into Ordinary Elliptic Curves\\
{\small Advances in Cryptology~--– CRYPTO 2010, 
Lecture Notes in Computer Science, 6223: 237--254, Springer-Verlag, 
2010}\\
%
{\small (Эфективное неразличимое хэширование на обычные эллиптические 
кривые)}

\bibitem{UTF8}
ISO/IEC 10646:2012
Information technology~-- Universal Coded Character Set (UCS)\\
{\small International Organization for Standardization, 2012}\\
{\small (Информационные технологии. 
Универсальный набор кодированных символов (UCS))}

\bibitem{DH}
Diffie~W., Hellman~M. 
New Directions in Cryptography\\
{\small IEEE Transactions on Information Theory, IT-22, 644–654, 1976}\\
{\small (Новые направления в криптографии)}

\label{LastBib}
\end{thebibliography}



