\begin{appendix}{А}{справочное}
{Протокол Диффи~--- Хеллмана}
\label{DH}

\mbox{}

Протокол Диффи -- Хеллмана, введенный в работе~\cite{DH}, 
неявно используется в протоколах BMQV, BSTS, BPACE
и является базовым для них.

Редакция протокола Диффи~--- Хеллмана, соответствующая соглашениям
настоящего стандарта, имеет следующий вид:
\begin{enumerate}
\item
Стороны~$A$ и~$B$ выбирают параметры эллиптической кривой
$p$, $a$, $b$, $q$ и~$G$, 
которые удовлетворяют требованиям, определенным в~\ref{COMMON.Params}. 

\item
Сторона~$A$ генерирует одноразовый личный ключ 
$u_A$
и определяет соответствующий открытый ключ~$V_A=u_A G$.
%
Аналогично,                
сторона~$B$ генерирует одноразовый личный ключ~$u_B$ 
и определяет открытый ключ~$V_B=u_B G$.
%
Личные ключи должны генерироваться 
в соответствии с требованиями, определенными 
в~\ref{COMMON.Ephemeral}.

\item
Стороны обмениваются открытыми ключами $V_A$, $V_B$.
%
Каждая из сторон проверяет, что полученный ею
открытый ключ является элементом $E_{a,b}^*(\FF_p)$,
и завершает протокол с ошибкой,
если проверяемое условие нарушается.
%
Для контроля~$V_A,V_B$ может использоваться
алгоритм проверки открытых ключей, 
определенный в СТБ~34.101.45 (пункт 6.2.3).

\item
Сторона~$A$ определяет точку~$K=u_A V_B$,
а сторона~$B$~--- точку $K=u_B V_A$.
Точки сторон совпадают: $u_A V_B = u_A u_B G = u_B V_A$.

\item
По общей точке~$K$ стороны строят общий секретный ключ,
используя, например, алгоритм~\ref{KDF.Alg}.
\end{enumerate}

%Реализация предпоследнего этапа протокола Диффи~--- Хеллмана сводится 
%к выполнению сторонами алгоритма, входными данными которого
%являются параметры эллиптической кривой, личный ключ~$u$ одной стороны
%и открытый ключ~$V$ противоположной стороны, а выходными~---
%общая точка~$uV$.

Пары ключей $(u_A,V_A)$, $(u_B,V_B)$
могут быть как одноразовыми (их еще называют эфемерными,
ephemeral), так и долговременными (статическими, static).
%
Соответственно может быть три варианта протокола 
Диффи~--- Хеллмана:
с одноразовыми ключами (ephemeral-ephemeral),
с долговременными ключами (static-static),
с долговременными и одноразовыми ключами (static-ephemeral).

На долговременные и одноразовые ключи протокола Диффи~--- Хеллмана
должны распространяться требования, заданные в~\ref{COMMON.Static}, 
\ref{COMMON.Ephemeral}.

В сеансах протокола с повторяющимися долговременными ключами 
общая точка~$K$ будет также повторяться. 
Для защиты от повторов общего ключа стороны 
должны использовать при его построении 
дополнительные уникальные (открытые или секретные) данные. 

Злоумышленник, который не вступает во взаимодействие со сторонами, 
а только перехватывает их сообщения, получает в свое распоряжение
параметры эллиптической кривой и открытые
ключи~$u_A G$ и~$u_B G$. Злоумышленнику требуется по этим 
данным определить общую секретную точку $u_A u_B G$. 
Такая задача называется вычислительной задачей Диффи~---Хеллмана
и считается трудной, если выбраны надежные параметры эллиптической кривой
(см.~\cite[пункт~4.1.5]{ECC}).

С другой стороны, протокол Диффи --- Хеллмана 
не защищает от атак злоумышленника <<посередине>>,
который выполняет протокол с каждой из сторон 
по-отдельности, выдавая себя за $A$ стороне~$B$ и за~$B$ стороне~$A$.
%
Злоумышленник формирует общие ключи с каждой из сторон,
а затем полностью контролирует обмен сообщений между ними,
перешифровывая данные или пересчитывая имитовставки.

Для защиты от злоумышленника <<посередине>> протокол
Диффи~-- Хеллмана должен сопровождаться дополнительными
механизмами безопасности. 
Фактически протокол Диффи~--- Хеллмана должен являться
составной частью некоторого высокоуровневого протокола,
который эти механизмы поддерживает.

Могут использоваться следующие два механизма, 
обеспечивающие надежное применение описанной выше 
редакции протокола Диффи~--- Хеллмана.

Первый механизм состоит в контроле целостности и подлинности 
открытых ключей при их передаче между сторонами.
%
Для организации контроля 
долговременные открытые ключи протокола могут 
передаваться в виде сертификатов.
%
Могут также использоваться долговременные ключи ЭЦП, с помощью которых стороны
вырабатывают и проверяют подписи данных обмена. (Именно такой подход использован
в протоколе BSTS).
%                               
Наконец для организации контроля могут использоваться 
долговременные секретные ключи имитозащиты,
с помощью которых стороны вырабатывают и проверяют 
имитовставки данных обмена.

Второй механизм состоит в использовании дополнительных секретных
данных при построении общего ключа.
%
Имеется в виду, что на вход алгоритма построения ключа кроме 
общей секретной точки~$K$ подаются секретные ключи,
предварительно распределенные или полученные в результате 
выполнения других протоколов.
%

\end{appendix}

\mbox{}
\vfill
\mbox{}
\clearpage
