\clearpage
\chapter*{\mbox{}\hfill Поправка к официальной редакции\hfill\mbox{}}

\mbox{}

\begin{center}
\begin{tabular}{|p{2.9cm}|p{6.2cm}|p{6.4cm}|}
\hline
В каком месте & Напечатано & Должно быть\\
\hline
\hline
Подраздел~\ref{AUX}
&
{\bf Алгоритм $\algname{belt-cfb}^{-1}$}.
В протоколе BPACE используется алгоритм расшифрования\ldots
&
{\bf Алгоритм $\algname{belt-cfb}^{-1}$}.
В протоколе BSTS используется алгоритм расшифрования\ldots 
\\
\hline
Пункт~\ref{MQV.Steps},\par
шаг 4.4
&
$t\leftarrow\langle\algname{belt-hash}
(\langle V_A\rangle_{2\ell}\parallel\langle V_B\rangle_{2\ell})\rangle$;
&
$t\leftarrow\langle\algname{belt-hash}
(\langle V_A\rangle_{2\ell}\parallel\langle V_B\rangle_{2\ell})\rangle_\ell$;
\\
\hline
Пункт~\ref{STS.Steps},\par
шаг 4.7
&
проверяет, что\par
$T_A\leftarrow\algname{belt-mac}(Y_A\parallel 0^{128}, K_1)$;
&
проверяет, что\par
$T_A=\algname{belt-mac}(Y_A\parallel 0^{128}, K_1)$;
\\
\hline
Приложение~\ref{DH},\par
предпоследний абзац
&
Могут также использоваться долговременые 
ключи ЭЦП, с помощью которых стороны вырабатывают и проверяют
подписи данных обмена.
&
Могут также использоваться долговременные 
ключи ЭЦП, с помощью которых стороны вырабатывают и проверяют
подписи данных обмена.
\\
\hline
\end{tabular}
\end{center}