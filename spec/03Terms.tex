\chapter{Термины и определения}

В настоящем стандарте применяют
следующие термины с соответствующими определениями:

{\bf \thedefctr~аутентификация}:
Проверка подлинности стороны.

{\bf \thedefctr~долговременный ключ}:
Ключ, который используется в нескольких сеансах протокола.

{\bf \thedefctr~зашифрование}:
Преобразование сообщения,
направленное на обеспечение его конфиденциальности,
которое выполняется с использованием секретного ключа.

{\bf \thedefctr~имитовставка}:
Контрольная характеристика сообщения, 
которая определяется с использованием секретного ключа 
и служит для контроля целостности и подлинности сообщения.

{\bf \thedefctr~имитозащита}:
Контроль целостности и подлинности сообщений, 
который реализуется путем выработки и проверки имитовставок.

{\bf \thedefctr~ключ}:
Параметр, который управляет криптографическими 
операциями зашифрования и расшифрования, 
имитозащиты, выработки и проверки ЭЦП, 
формирования общего ключа и др.

{\bf \thedefctr~конфиденциальность}:
Гарантия того, что сообщения доступны для использования
только тем сторонам, которым они предназначены.

{\bf \thedefctr~личный ключ:}
Ключ, который связан с конкретной стороной, не является общедоступным и
используется в настоящем стандарте для формирования общего ключа и для выработки
электронной цифровой подписи.

%{\bf \thedefctr~общий ключ}:
%Секретный ключ, 
%распределенный между сторонами в результате 
%формирования общего ключа.

{\bf \thedefctr~одноразовый ключ}:
Ключ, который создается, используется и уничтожается в течение одного 
сеанса протокола.

{\bf \thedefctr~октет}:
Двоичное слово длины~$8$.

{\bf \thedefctr~открытый ключ:}
Ключ, который строится по личному ключу, связан с конкретной стороной, 
может быть сделан общедоступным и используется в настоящем стандарте 
для формирования общего ключа и для проверки 
электронной цифровой подписи.

{\bf \thedefctr~пароль}:
Секрет, который способен запомнить человек и 
который поэтому может принимать сравнительно небольшое 
число значений.
            
%This part of ISO/IEC 11770 defines key establishment mechanisms 
%based on weak secrets, i.e., secrets that 
%can be readily memorized by a human, and hence secrets that 
%will be chosen from a relatively small set of
%possibilities.

%secret that can be conveniently memorized by a human being; 
%typically this means that the entropy of the
%secret is limited, so that an exhaustive search 
%for the secret may be feasible, given knowledge that would
%enable a correct guess for the secret to be 
%distinguished from an incorrect guess

{\bf \thedefctr~подлинность}:
Гарантия того, что сторона действительно та, за кого себя выдает; гарантия того,
что сторона действительно является владельцем (создателем, отправителем)
определенного сообщения.

{\bf \thedefctr~подтверждение ключа}:
Проверка того, что сформированый стороной общий ключ корректен.

{\bf \thedefctr~построение ключа}:
Создание ключа по общим секретным и дополнительным открытым данным, завершающая
стадия формирования общего ключа.

{\bf \thedefctr~протокол}:
Интерактивный криптографический алгоритм, который выполняют несколько
сторон-участников, обмениваясь между собой сообщениями, содержащими
промежуточные результаты вычислений.

{\bf \thedefctr~расшифрование}:
Преобразование, обратное зашифрованию.

{\bf \thedefctr~сеанс протокола}:
Конкретная реализация (прогон) протокола.

{\bf \thedefctr~секретный ключ}:
Ключ, который связан с конкретными сторонами,
не является общедоступным и используется в настоящем стандарте 
для шифрования, имитозащиты, формирования общего ключа,
генерации псевдослучайных чисел.

{\bf \thedefctr~сертификат}:
Структурированные данные, связывающие 
идентификатор стороны с ее долговременным открытым ключом.

{\bf \thedefctr~синхропосылка}:
Открытые входные данные криптографического алгоритма,
которые обеспечивают уникальность результатов 
криптографического преобразования на фиксированном ключе.

{\bf \thedefctr~сообщение}:
Двоичное слово конечной длины.

{\bf \thedefctr~формирование общего ключа}:
Процедура, в результате которой несколько сторон
определяют один и тот же ключ, известный только им.

{\bf \thedefctr~хэш-значение}:
Двоичное слово фиксированной длины, 
которое определяется по сообщению без использования ключа и 
служит для контроля целостности сообщения и для представления 
сообщения в (необратимо) сжатой форме.

{\bf \thedefctr~хэширование}:
Выработка хэш-значений.

{\bf \thedefctr~целостность}:
Гарантия того, что сообщение не изменено 
при хранении или передаче.

{\bf \thedefctr~шифрование}:
Зашифрование или расшифрование.

{\bf \thedefctr~электронная цифровая подпись; ЭЦП}:
Контрольная характеристика сообщения,
которая вырабатывается с использованием личного ключа,
проверяется с использованием открытого ключа,
служит для контроля целостности и подлинности сообщения
и обеспечивает невозможность отказа от авторства.

