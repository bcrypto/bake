\section{Построение ключа}\label{KDF}

\subsection{Входные и выходные данные}

Входными данными алгоритма построения ключа являются
секретное слово~$X\in\{0,1\}^*$, 
дополнительное слово~$S\in\{0,1\}^*$
и номер ключа~$C$~--- неотрицательное целое число.

Выходными данными является ключ~$Y\in\{0,1\}^{256}$.

\subsection{Вспомогательные алгоритмы}\label{KDF.Aux}

{\bf Алгоритм~\algname{belt-hash}}.
Используется алгоритм хэширования~\algname{belt-hash},
определенный в СТБ~34.101.31 (пункт~6.9.3).
Входными данными алгоритма является слово~$X\in\{0,1\}^*$,
выходными~--- его хэш-значение~$Y\in\{0,1\}^{256}$.

{\bf Алгоритм~\algname{belt-keyrep}}.
Используется алгоритм преобразования ключа~\algname{belt-keyrep},
определенный в СТБ~34.101.31 (пункт~7.2.3).
%
Входными данными алгоритма является 
преобразуемый ключ~$X\in\{0,1\}^{256}$,
уровень~$D\in\{0,1\}^{96}$, заголовок~$I\in\{0,1\}^{128}$,
длина~$m\in\{128,192,256\}$.
%
Выходными данными является 
преобразованный ключ~$Y\in\{0,1\}^m$.

\subsection{Алгоритм построения ключа}\label{KDF.Alg}

Построение ключа состоит в выполнении следующих шагов:
\begin{enumerate}
\item
$Y\leftarrow\algname{belt-hash}(X\parallel S)$.

\item
$Y\leftarrow\algname{belt-keyrep}(Y,1^{96},\langle C\rangle_{128},256)$.

\item
Возвратить~$Y$.
\end{enumerate}
