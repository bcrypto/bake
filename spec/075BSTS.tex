\section{Протокол BSTS}\label{STS}

\subsection{Сообщения}\label{STS.Messages}

В протоколе BSTS
стороны пересылают друг другу следующие сообщения:

M0 ($A\to B$): 
$\mathimplicit{\hello_A}$;

M1 ($B\to A$): 
$\mathimplicit{\hello_B\parallel}\ 
\langle V_B\rangle_{4\ell}$;

M2 ($A\to B$): 
$\langle V_A\rangle_{4\ell}\parallel Y_A\parallel T_A$;

M3 ($B\to A$): 
$Y_B\parallel T_B$.

\subsection{Шаги}\label{STS.Steps}

Формирование общего ключа состоит в выполнении шагов, определенных ниже. При
ошибке на любом из шагов, в том числе при отрицательном результате любой
проверки, протокол прекращается и возвращается признак~\texttt{ОШИБКА}.

\begin{enumerate}
\item
Сторона $A$:
\begin{enumerate}
\item
\implicit{отправляет сообщение M0}.
\end{enumerate}

\item
Сторона~$B$:
\begin{enumerate}
\item
\implicit{получает сообщение M0};
\item
$u_B\stackrel{R}{\leftarrow}\{1,2,\ldots,q-1\}$
(в соответствии с требованиями~\ref{COMMON.Ephemeral});
\item
$V_B\leftarrow u_B G$;
\item
отправляет сообщение M1.
\end{enumerate}

\item
Сторона $A$:
\begin{enumerate}
\item
получает сообщение M1;
\item
проверяет, что $V_B\in E^*(\FF_p)$;
\item
$u_A\stackrel{R}{\leftarrow}\{1,2,\ldots,q-1\}$
(в соответствии с требованиями~\ref{COMMON.Ephemeral});
\item
$V_A\leftarrow u_{A} G$;
\item
$K\leftarrow u_A V_B$;
\item
$K_0\leftarrow\algname{bake-kdf}
(\langle K\rangle_{2\ell},\hello_A\parallel\hello_B,0)$;
\item
$K_1\leftarrow\algname{bake-kdf}
(\langle K\rangle_{2\ell},\hello_A\parallel\hello_B,1)$;
\item
$K_2\leftarrow\algname{bake-kdf}
(\langle K\rangle_{2\ell},\hello_A\parallel\hello_B,2)$;
\item
$t\leftarrow\bigl\langle\algname{belt-hash}(
\langle V_A\rangle_{2\ell}\parallel\langle V_B\rangle_{2\ell})
\bigr\rangle_\ell$;
\item
$s_A\leftarrow (u_A-(2^\ell +\overline{t})d_A)\bmod q$;
\item
$Y_A\leftarrow\algname{belt-cfb}
(\langle s_A\rangle_{2\ell}\parallel\Cert(Id_A,Q_A),K_2,0^{128})$;
\item
$T_A\leftarrow\algname{belt-mac}(Y_A\parallel 0^{128}, K_1)$;
\item
отправляет сообщение M2.
\end{enumerate}

\item
Сторона $B$:
\begin{enumerate}
\item
получает сообщение M2;
\item
проверяет, что $V_A\in E^*(\FF_p)$;
\item
$K\leftarrow u_B V_A$;
\item
$K_0\leftarrow\algname{bake-kdf}
(\langle K\rangle_{2\ell},\hello_A\parallel\hello_B,0)$;
\item
$K_1\leftarrow\algname{bake-kdf}
(\langle K\rangle_{2\ell},\hello_A\parallel\hello_B,1)$;
\item
$K_2\leftarrow\algname{bake-kdf}
(\langle K\rangle_{2\ell},\hello_A\parallel\hello_B,2)$;
\item
проверяет, 
что $T_A=\algname{belt-mac}
(Y_A\parallel 0^{128}, K_1)$;
\item
$\langle s_A\rangle_{2\ell}\parallel\Cert(Id_A,Q_A)
\leftarrow\algname{belt-cfb}^{-1}(Y_A,K_2,0^{128})$;
\item
проверяет, что $s_A\in\{0,1,\ldots,q-1\}$;
\item
проверяет $\Cert(Id_A,Q_A)$;
\item
$t\leftarrow\bigl\langle\algname{belt-hash}
(\langle V_A\rangle_{2\ell}\parallel \langle V_B\rangle_{2\ell})
\bigr\rangle_\ell$;
\item
проверяет, что $s_A G+(2^\ell +\overline{t})Q_A=V_A$;
\item
$s_B\leftarrow (u_B-(2^\ell+\overline{t})d_B)\bmod q$;
\item
$Y_B\leftarrow\algname{belt-cfb}
(\langle s_B\rangle_{2\ell}\parallel\Cert(Id_B,Q_B),K_2,1^{128})$;
\item
$T_B\leftarrow\algname{belt-mac}
(Y_B\parallel 1^{128}, K_1)$;
\item
отправляет сообщение~M3.
\end{enumerate}

\item
Сторона $A$:
\begin{enumerate}
\item
получает сообщение M3;
\item
проверяет, 
что $T_B=\algname{belt-mac}
(Y_B\parallel 1^{128}, K_1)$;
\item
$\langle s_B\rangle_{2\ell}\parallel\Cert(Id_B,Q_B)
\leftarrow\algname{belt-cfb}^{-1}(Y_B,K_2,1^{128})$;
\item
проверяет, что $s_B\in\{0,1,\ldots,q-1\}$;
\item
проверяет $\Cert(Id_B,Q_B)$;
\item
проверяет, что $s_B G+(2^\ell +\overline{t})Q_B=V_B$.
\end{enumerate}
\end{enumerate}

Успешное выполнение всех шагов протокола означает, что 
стороны выработали общий ключ~$K_0$ 
и явно подтвердили его друг другу, 
в том числе провели взаимную аутентификацию.

