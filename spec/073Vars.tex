\section{Переменные}\label{VARS}

{\bf Одноразовые личные ключи}.
Используются одноразовые личные ключи~$u_A,u_B\in\{1,2,\ldots,q-1\}$. Требования
по управлению этими ключами определены в~\ref{COMMON.Ephemeral}.

{\bf Одноразовые открытые ключи}.
Используются одноразовые открытые ключи~$V_A,V_B\in E_{a,b}^*(\FF_p)$.
                                                  
{\bf Одноразовые секретные ключи}.
В протоколе BPACE используются одноразовые секретные
ключи~$R_A,R_B\in\{0,1\}^\ell$. Требования по управлению этими ключами
определены в~\ref{COMMON.Ephemeral}.

{\bf Одноразовые подписи}. 
В протоколах BMQV, BSTS вырабатываются одноразовые
подписи~$s_A,s_B\in\{0,1,\ldots,q-1\}$. В BMQV одноразовые подписи должны быть
уничтожены после использования.

{\bf Общая секретная точка}. 
Формируется общая секретная точка~$K\in E_{a,b}(\FF_p)$. Значение~$K$ должно
быть уничтожено после использования.

{\bf Имитовставки}. 
Для подтверждения общего~ключа~$K_0$ используются
имитовставки~$T_A,T_B\in\{0,1\}^{64}$.

{\bf Служебный ключ~$K_1$}. 
Для формирования имитовставок используется служебный ключ~$K_1\in\{0,1\}^{256}$.
Ключ~$K_1$ должен быть уничтожен после использования.

{\bf Служебный ключ~$K_2$}.                             
В протоколах~BSTS и BPACE для шифрования данных используется служебный
ключ~$K_2\in\{0,1\}^{256}$. Ключ~$K_2$ должен быть уничтожен после
использования.

{\bf Зашифрованные данные}. 
В протоколе BSTS с помощью~$K_2$ выполняется зашифрование одноразовых подписей и
сертификатов. В результате формируются слова~$Y_A,Y_B\in\{0,1\}^*$.
%
В протоколе BPACE зашифровываются одноразовые секретные ключи. В результате
формируются слова~$Y_A,Y_B\in\{0,1\}^\ell$.

{\bf Переменная~$t$}. 
В протоколах~BMQV и~BSTS каждая из сторон использует
переменную~$t\in\{0,1\}^\ell$.

{\bf Переменная~$W$}. 
В протоколе~BPACE каждая из сторон использует 
переменную~$W\in E_{a,b}^*(\FF_p)$.
%
Значение~$W$ должно быть уничтожено после использования.
