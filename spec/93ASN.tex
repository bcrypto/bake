\begin{appendix}{В}{рекомендуемое}
{Модуль АСН.1}
\label{ASN}

\hiddensection{Идентификаторы}\label{ASN.OIDs}

Алгоритмам и протоколам стандарта присваиваются 
следующие идентификаторы:
\begin{center}
\begin{tabular}{p{4cm}p{12cm}}
\texttt{bake-bmqv} &
протокол BMQV (\ref{MQV});\\
%
\texttt{bake-bsts} &
протокол BSTS (\ref{STS});\\
%
\texttt{bake-bpace} &
протокол BPACE (\ref{PACE});\\
%
\texttt{bake-dh} &
алгоритм определения общей точки по личному ключу 
одной стороны и открытому ключу другой стороны,
часть протокола Диффи~--- Хеллмана (приложение~\ref{DH});\\
%
\texttt{bake-kdf} &
алгоритм построения ключа (\ref{KDF.Alg});\\
%
\texttt{bake-swu} &
алгоритм построения точки эллиптической кривой (\ref{SWU.Alg}).\\
\end{tabular}
\end{center}

Уровень стойкости протоколов ключа не указывается 
в их идентификаторах и определяется по размерностям параметров 
используемой эллиптической кривой. 

Долговременному открытому ключу, 
который используется в протоколах BMQV и BSTS, 
присваивается идентификатор \texttt{bake-pubkey}.

\hiddensection{Описание долговременного открытого ключа}\label{ASN.PubKey}

На уровне стойкости~$\ell$ долговременному открытому ключу~$Q$ 
ставится в соответствие двоичное слово~$\langle Q\rangle_{4\ell}$,
для описания которого может использоваться тип \texttt{PublicKey},
определенный в СТБ~34.101.45 (приложение~Д).

В запросе на получение сертификата (СТБ~34.101.17),
в сертификатах (СТБ~34.101.19)
открытый ключ должен представляться значениями типа
\texttt{SubjectPublicKeyInfo}. 
Этот тип также определен в СТБ~34.101.45
(приложение~Д).

\hiddensection{Модуль АСН.1}\label{ASN.Module}

\verbatiminput{bake-module-v1.asn}

%  IMPORTS
%    bign-curves, bign-curve256v1, bign-curve384v1, bign-curve512v1, 
%    bign-fields, bign-primefield, AlgorithmIdentifier, DomainParameters,
%    ECParameters, FieldID, Curve, PublicKey, SubjectPublicKeyInfo
%    FROM Bign-module-v2 {iso(1) member-body(2) by(112) 0 2 0 34 101 
%         45 module(1) ver2(2)}

\end{appendix}