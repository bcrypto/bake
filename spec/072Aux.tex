\section{Вспомогательные алгоритмы}\label{AUX}

{\bf Алгоритм \algname{belt-ecb}}.
В протоколе BPACE используется алгоритм зашифрования 
в режиме простой замены~\algname{belt-ecb},
определенный в СТБ~34.101.31 (пункт~6.2.3).
Входными данными алгоритма являются сообщение~$X\in\{0,1\}^*$, 
длина которого не меньше~$128$, и
ключ~$\theta\in\{0,1\}^{256}$.
Выходными данными является 
зашифрованное сообщение~$Y\in\{0,1\}^{|X|}$.

{\bf Алгоритм $\algname{belt-ecb}^{-1}$}.
В протоколе BPACE используется алгоритм расшифрования 
в режиме простой замены~$\algname{belt-ecb}^{-1}$, 
определенный в СТБ~34.101.31 (пункт~6.2.4).
Входными данными алгоритма являются 
зашифрованное сообщение~$Y\in\{0,1\}^*$, 
длина которого не меньше~$128$,
и ключ~$\theta\in\{0,1\}^{256}$.
Выходными данными является 
первоначальное сообщение~$X\in\{0,1\}^{|Y|}$.

{\bf Алгоритм \algname{belt-cfb}}.
В протоколе BSTS используется алгоритм зашифрования 
в режиме гаммирования с обратной связью~\algname{belt-cfb},
определенный в СТБ~34.101.31 (пункт~6.4.3).
Входными данными алгоритма являются сообщение~$X\in\{0,1\}^*$, 
ключ~$\theta\in\{0,1\}^{256}$ и синхропосылка~$S\in\{0,1\}^{128}$.
Выходными данными является 
зашифрованное сообщение~$Y\in\{0,1\}^{|X|}$.

{\bf Алгоритм $\algname{belt-cfb}^{-1}$}.
В протоколе BSTS используется алгоритм расшифрования 
в режиме гаммирования с обратной связью~$\algname{belt-cfb}^{-1}$, 
определенный в СТБ~34.101.31 (пункт~6.4.4).
Входными данными алгоритма являются 
зашифрованное сообщение~$Y\in\{0,1\}^*$, 
ключ~$\theta\in\{0,1\}^{256}$ и синхропосылка~$S\in\{0,1\}^{128}$.
Выходными данными является 
первоначальное сообщение~$X\in\{0,1\}^{|Y|}$.

{\bf Алгоритм \algname{belt-mac}}.
Используется алгоритм выработки 
имитовставки~\algname{belt-mac}, 
определенный в СТБ~34.101.31 (пункт~6.6.3).
Входными данными алгоритма являются сообщение~$X\in\{0,1\}^*$
и ключ~$\theta\in\{0,1\}^{256}$,
выходными~--- имитовставка~$T\in\{0,1\}^{64}$.

{\bf Алгоритм~\algname{belt-hash}}.
Используется алгоритм хэширования~\algname{belt-hash},
описанный в~\ref{KDF.Aux}.

{\bf Алгоритм \algname{bake-kdf}}.
Используется алгоритм построения ключа~\algname{bake-kdf},
определенный в~\ref{KDF.Alg}.

{\bf Алгоритм \algname{bake-swu}}.
В протоколе BPACE используется алгоритм построения точки 
эллиптической кривой~\algname{bake-swu},
определенный в~\ref{SWU.Alg}.

{\bf Проверка открытого ключа}.
Стороны проверяют, что присланные открытые 
ключи~$V_A$, $V_B$ являются аффинными точками эллиптической кривой.
Контроль может быть выполнен с помощью алгоритма
проверки открытого ключа, определенного в СТБ~34.101.45 
(пункт~6.2.3).

